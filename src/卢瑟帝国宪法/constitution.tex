\documentclass[UTF8,11pt,a4paper]{ctexart} 
\usepackage{graphicx} 
\usepackage{float}
\usepackage{geometry}
\usepackage{xcolor}
\usepackage{fancyhdr}
\usepackage[colorlinks,linkcolor=blue]{hyperref}
\includeonly{loser_empire.tex}
\geometry{left=2cm,right=2cm,top=3cm,bottom=2cm}
\pagestyle{fancy}
\lhead{\author}
\chead{\date}
\rhead{1 你好卢瑟}
\lfoot{}
\cfoot{\thepage}
\rfoot{}
\renewcommand{\headrulewidth}{0.2pt}
\renewcommand{\headwidth}{\textwidth}
\renewcommand{\footrulewidth}{0pt}
\title{卢瑟帝国宪法 }
\author{mq白}
\date{\today}
\begin{document}
\maketitle
\section{你好 卢瑟}
事实上在阅读宪法之前,请先明确自己的身份,是“\textbf{卢瑟}”,否则则不需要遵守。
不管什么职位,都是 \textbf{卢瑟}。
\subsection{你好 卢瑟王}
\paragraph{卢瑟王}是卢瑟帝国的王者,可以统御普通卢瑟。
\subsection{底层卢瑟你好}
\paragraph{国王}在卢瑟帝国有统治权。\textbf{mq白}是国王,管理一切, 是卢瑟帝国中心。
\subsection{你好 卢瑟管理者}
\paragraph{卢瑟管理者}是专门管理卢瑟的,地位和卢瑟王一样。

\subsection{你好 赞助商}
\paragraph{赞助商}是卢瑟帝国唯一的资金来源,非常重要,用于供养国王和帝国日常发展。\\
{\color{white}-}
\begin{figure}[H]
    \caption{简单等级示意图} 
    \centering 
    \includegraphics[width = .7\textwidth]{resource/luse.png}
\end{figure}

\begin{center}
    \begin{tabular}{|l|c|r|}
        \hline
             & \textbf{卢瑟帝国国王} &       \\
        \hline
        卢瑟王  & 赞助商             & 卢瑟管理者 \\
        \hline
        mq家族 & 史官              & 底层卢瑟  \\
        \hline
             & 新子民             &       \\
        \hline
    \end{tabular}
\end{center} 

\section{如何做一名卢瑟?}

这应该是一个很好的问题,也是大家都会思考的问题。

关于如何做好一名 \textbf{卢瑟},我认为有以下 \textbf{3}点。

\subsection{技能}

卢瑟至少应该掌握以下基本技能:

\begin{enumerate}

    \item 熟练使用 \href{https://zh.cppreference.com/w/cpp}{cppreference} 文档。
    \item 熟练使用现代 C++。
    \item 学习过操作系统,计算机网络 等基本课程。
    \item 能够使用 Linux 和 windows 系统 api 进行编程。
    \item 学习过至少 4 门语言。
    \item 使用了解过常见的 GUI 框架(如 Qt,WinForms,mfc,wpf)。
    \item git 的基本使用。
\end{enumerate}

\subsection{互助}

人人为我,我为人人,在有时间的时候热情的回答和思考其他卢瑟的问题。

\subsection{包容}

包容卢瑟们的怪癖,人无完人。



\section{如何成为一名高级卢瑟?}

\begin{enumerate}

    \item 独立完成 \href{https://github.com/Mq-b/Loser-HomeWork}{Loser\_Homewok} 全部 13 道题目。
    \item 为 \href{https://github.com/Mq-b/Loser-HomeWork}{Loser\_Homewok} 做贡献(提 pr 或 Issues)。
    \item 有足够的利用价值。
\end{enumerate}

\textbf{如果能在必要时帮助国王,那么以上一切条件都不需要}。

\end{document}