\documentclass{ctexart}
\usepackage{graphicx} 
\usepackage{float}
\usepackage{geometry}
\usepackage{xcolor}
\usepackage{fancyhdr}
\usepackage{minted}
\begin{document}

\begin{minted}{c}
int main() {
    printf("hello, world");
return 0;
}
\end{minted}

\mint{c++}|C++20coroutine_/test.cpp|
\begin{minted}[mathescape,	% 中括号中的内容用于控制代码显示的格式,可以依照喜好修改
    linenos,
    numbersep=5pt,
    gobble=2,
    frame=lines,
    framesep=2mm]{c++}
    
    #include <coroutine>
    #include <iostream>
    
    struct promise {
        struct promise_type;
    
        struct iterator {
            std::coroutine_handle<promise_type>& h;
            int& operator*() {
                return h.promise().n;
            }
            iterator operator++() {
                if (!h.done())h.resume();
                return *this;
            }
            bool operator!=(const iterator&)const {
                return !h.done();
            }
        };
        struct promise_type {
            int n;
            promise get_return_object() {
                return { std::coroutine_handle<promise_type>::from_promise(*this) };
            }
            std::suspend_never initial_suspend() noexcept { return {}; }//注意返回类型 需要确保协程第一次进去被执行了
            std::suspend_always final_suspend() noexcept { return {}; }
            std::suspend_always yield_value(int r) { n = r; return {}; }//co_yield()需要
            void return_void() { }
            void unhandled_exception() {}
        };
        iterator begin() { return { _h }; }
        iterator end() { return { _h }; }
        std::coroutine_handle<promise_type>_h;
    };
    
    promise iota(int value) {
        std::cout << "iota\n";
        while (value) {
            co_yield value;
            --value;
        }
    }
    
    int main() {
        for (int x : iota(10)) {//范围for会执行operator!= * ++ 改变协程状态写到++中,*普通的返回值就好 
            std::cout << x << '\n';
        }
    }
    
\end{minted}

行内代码\mintinline{python}{import numpy as np}

\end{document}

% pdflatex -shell-escape  test.tex