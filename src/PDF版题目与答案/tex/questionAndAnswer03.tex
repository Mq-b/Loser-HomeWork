日期:2023/7/24 出题人:mq白\\

实现一个 print,如果你做了上一个作业,我相信这很简单。 要求调用形式为:

\begin{minted}[mathescape,	
    linenos,
    numbersep=5pt,
    gobble=2,
    frame=lines,
    framesep=2mm]{c++}
    print(格式字符串, 任意类型和个数的符合格式字符串要求的参数)
\end{minted}

\begin{minted}[mathescape,	
    linenos,
    numbersep=5pt,
    gobble=2,
    frame=lines,
    framesep=2mm]{c++}
    struct Frac {
        int a, b;
     };
\end{minted}

给出自定义类型Frace,要求支持以下:

\begin{minted}[mathescape,	
    linenos,
    numbersep=5pt,
    gobble=2,
    frame=lines,
    framesep=2mm]{c++}
    Frac f{ 1,10 };
    print("{}", f);// 结果为1/10
\end{minted}

\begin{tcolorbox}[title = {要求运行结果},
        fonttitle = \bfseries, fontupper = \sffamily, fontlower = \itshape]
    1/10
\end{tcolorbox}

\begin{itemize}
    \item \textbf{难度}: \hardscore{3} \\
          \textbf{提示}:std::formatter。
\end{itemize}

禁止面向结果编程,使用宏等等方式,本题主要考察和学习 format 库,记得测试至少三个不同编译器。


\subsection{标准答案}

\begin{minted}[mathescape,	
    linenos,
    numbersep=5pt,
    gobble=2,
    frame=lines,
    framesep=2mm,
    breaklines]{c++}
    template<>
    struct std::formatter<Frac>:std::formatter<char>{
        auto format(const auto& frac, auto& ctx)const{//const修饰是必须的
            return std::format_to(ctx.out(), "{}/{}", frac.a, frac.b);
        }
    };
    void print(std::string_view fmt,auto&&...args){
        std::cout << std::vformat(fmt, std::make_format_args(std::forward<decltype(args)>(args)...));
    }
\end{minted}

\subsection{解析}

实现一个 print 很简单,我们只要按第二题的思路来就行了,一个格式化字符串,用 std::string\_view 做第一个形参,另外需要接受任意参数和个数的参数,使用形参包即可。

\begin{minted}[mathescape,	
    linenos,
    numbersep=5pt,
    gobble=2,
    frame=lines,
    framesep=2mm,
    breaklines]{c++}
    void print(std::string_view fmt,auto&&...args){
        std::cout << std::vformat(fmt, std::make_format_args(std::forward<decltype(args)>(args)...));
    }
\end{minted}

由于使用了 C++20 简写函数模板,此处的完美转发就只能使用 decltype,显得有点诡异,其实很合理

展开的形式无非就是:

\begin{minted}[mathescape,	
    linenos,
    numbersep=5pt,
    gobble=2,
    frame=lines,
    framesep=2mm,
    breaklines]{c++}
    std::forward<decltype(args1)>(args1),
    std::forward<decltype(args2)>(arsg2),
    std::forward<decltype(args3)>(args3),... 
\end{minted}

这两个格式化函数没必要再介绍,和第第二题的作用一样,很简单的调库。

\clearpage